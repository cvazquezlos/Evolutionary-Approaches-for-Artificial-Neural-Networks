\chapter{Contenidos del TFM}
Durante la elaboración de la memoria del Trabajo Fin de Máster, el alumno deberá incluir toda la información que considere necesaria y útil para la descripción y justificación del trabajo desarrollado y de los resultados obtenidos. Además y con objeto de asegurar que ha cubierto todas las competencias transversales asociadas al mismo, es obligatorio que en los diferentes apartados de la memoria se desarrollen cuidadosamente los aspectos indicados a continuación.

\paragraph{Introducción y Objetivos:}
Justificar la necesidad de desarrollar la tesis en lugar de adquirir o aplicar directamente lo existente a lo largo de la gama de categorías de procesos, productos y servicios de la empresa o institución usuaria de la tesis.

\paragraph{Estado del arte:}
Demostrar que se ha comprendido como es el ámbito de negocio donde se enmarca la tesis, sus hábitos y necesidades de productos o servicios tecnológicos.

\paragraph{Evaluación de Riesgos:}
Demostrar que se han considerado diferentes soluciones tanto clásicas como novedosas o innovadoras al problema, y se han evaluado los riesgos y ventajas de cada una de ellas.

\paragraph{Resultados:}
Justificar que la tecnología resultante de la tesis satisface los deseos o necesidades del cliente (real, potencial o ficticio).

\paragraph{Conclusiones:}
Establecer las conclusiones del trabajo apoyándose fundamentalmente en los datos y observaciones obtenidas durante su desarrollo.
Discutir que medios, cauces, etapas y tecnologías harían falta (si procede) para llevar a cabo una implantación real de los resultados.

\paragraph{Líneas futuras:}
Discutir los límites de las tecnologías actuales aplicadas al problema, planteando líneas de I+D+i realistas y capaces de superarlos.