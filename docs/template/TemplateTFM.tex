\documentclass[spanish,12pt,twoside]{book}
\usepackage{newclude} % Permite usar \include* que es igual que \include pero sin saltos de página adicionales
%\includeonly{cap01} % con este comando solo se incluirán los archivos mencionados haciendo que la compilación sea mucho más rápida. Referencias a otros capítulos quedarán rotas, se recomienda una compilación limpia cada vez que se cambie esta línea (es decir, borrar ficheros intermedios cuando se descomenta/comenta/cambia el argumento)

\usepackage[top=2.5cm, bottom=2.5cm,left= 4cm,right=2cm]{geometry}

\usepackage[es-tabla,es-nodecimaldot]{babel} % Cambiamos a idioma español
\usepackage[utf8]{inputenc} % Decimos que espere codificación utf-8 de los ficheros
\usepackage{csquotes} % Paquete recomendado por biblatex cuando se usa babel
\usepackage[T1]{fontenc}

\usepackage[hidelinks]{hyperref} % Referencias e índices son enlaces al contenido. La opción hidelinks hace que se vean como texto normal

\usepackage{graphicx} % Gráficos y colores
\usepackage{amsmath,amssymb}
\usepackage{float}
\usepackage{changepage}
\usepackage{subcaption}

\usepackage[table]{xcolor} % Necesaria la opción table para usar filas de colores en las tablas.
\usepackage{lipsum} % Para usar texto de relleno, sirve para tener una idea de cómo queda

\usepackage{booktabs} % para los comandos toprule, midrule, cmidrule y bottomrule: líneas horizontales de las tablas
\usepackage{threeparttable} % Entorno para las notas a pie de tabla

%--- Paquete algorithm y personalizaciones ---%
\usepackage{algorithm,algorithmic}
\floatname{algorithm}{Altoritmo}
\renewcommand{\listalgorithmname}{Índice de algoritmos}

\usepackage{multirow}

\usepackage{tikz}

%--- Formato de títulos ---%
\usepackage{titlesec} % Librería para cambiar el formato de \chapter, \section, ...
\titleformat{\paragraph}[runin]{\normalfont\normalsize\itshape}{\theparagraph}{1em}{} % formato de \paragraph: fuente y tamaño normales y cursiva. 1em de separación
\titleformat{\subparagraph}[runin]{\normalfont\normalsize\itshape}{\thesubparagraph}{1em}{} % formato de \subparagraph: fuente y tamaño normales y cursiva. 1em de separación

%--- AJUSTES MANUALES ---%

% Definimos un nuevo entorno para tablas con filas alternando colores:
%  - es necesario cargar el paquete xcolor con la opción table
%  - Usar los comandos \hiderowcolors y \showrowcolors cuando se quiera parar o empezar a usar la alternancia de colores dentro de una tabla
\newenvironment{stripedtable}
	{\rowcolors{2}{gray!6}{white}\begin{table}}
	{\end{table}\rowcolors{2}{}{}}

%%%%%%%%%%%%%%%%%%%%%%%%%%%%% DOCUMENT %%%%%%%%%%%%%%%%%%%%%%%%%%%%%
\begin{document}

\frontmatter
%%%%%%%%%%%%%%%%%%%%%%%%%%%%% TÍTULO, RESUMEN, AGRADECIMIENTOS Y LISTAS %%%%%%%%%%%%%%%%%%%%%%%%%%%%%
%	TÍTULO
% Página con el título, hay que entrar en title.tex para modificar: nombre del trabajo, autor, tutores y fecha
\include*{title}

%	AGRADECIMIENTOS Y RESUMEN
\include*{agradecimientos_y_resumen}

%	ÍNDICE
\tableofcontents % indice de contenidos

%	INDICE DE FIGURAS, TABLAS Y ALGORITMOS
\listoffigures
\listoftables
\listofalgorithms % Si no hay muchos algoritmos no tiene mucho sentido esta lista

%%%%%%%%%%%%%%%%%%%%%%%%%%%%% CUERPO DE LA TFM %%%%%%%%%%%%%%%%%%%%%%%%%%%%%
\mainmatter

% TABLAS, FIGURAS, EXPRESIONES MATEMÁTICAS Y ALGORITMOS
\include*{cap01}
% CONTENIDOS DE LA TFM
\include*{cap02}
% CONCLUSIONES Y LÍNEAS FUTURAS DE TRABAJO
\include*{cap03}
% SOBRE LAS REFERENCIAS
\include*{cap04}
% ESQUELETO DE UN CAPÍTULO
\include*{cap05}

%%%%%%%%%%%%%%%%%%%%%%%%%%%%% APÉNDICES %%%%%%%%%%%%%%%%%%%%%%%%%%%%%
\appendix
% ANEXOS
\include*{anexos}

%%%%%%%%%%%%%%%%%%%%%%%%%%%%% BACKMATTER %%%%%%%%%%%%%%%%%%%%%%%%%%%%%
\backmatter

%%% Bibliografía
\addcontentsline{toc}{chapter}{Bibliografía} % Para que salga en el índice general
\begin{thebibliography}{00}
	\bibitem{Ashtiani2014}  Ashtiani, M.H.Z., Ahmadabadi, M.N., Araabi, B.N. (2014). Bandit-based local feature subset selection. \emph{Neurocomputing} 138, 371--382.
	\bibitem{Berry1985} Berry, D., Fristedt, B. (1985). \emph{Bandit problems}. London: Chapman and Hall.
	\bibitem{Figueira2005} Figueira, J., Mousseau, V., Roy, B. (2005). Electre methods. En J. Figueira, S. Greco y M. Erghott (Eds.), \emph{Multiple criteria decision analysis. State of the art survey} (pp. 133--162). New York: Springer.
	\bibitem{Li2010} Li, L., Chu, W., Langford, J., Schapire, R.E. (2010). A contextual-bandit approach to personalized news article recommendation. En \emph{Proceedings of the 19th International Conference on World Wide Web} (pp. 661--670). New York: ACM.
	\bibitem{Mateos2009} Mateos, A., Jiménez, A. (2009). A trapezoidal fuzzy numbers-based approach for aggregating group preferences and ranking decision alternatives in MCDM. En M. Erghott, C.M. Fonseca, X. Gandibleux, H. Jao y M. Servaux (Eds.). \emph{Evolutionary multi-criterion optimization} (pp. 365--379). Berlin: Springer.
	\bibitem{Sutton1998} Sutton, R. Barto, A. (1998). \emph{Reinforcement learning, an introduction}. Cambridge: MIT Press.
	\bibitem{Thompson1933} Thompson, W.R. (1933). On the likelihood that one unknown probability exceeds another in view of the evidence of two samples. \emph{Biometrika} 25(3-4), 285--294.
	\bibitem{Vicente2016} Vicente, E. (2016). \emph{Análisis y gestión del riesgo en los sistemas de información: Un enfoque borroso}. (Tesis doctoral). Universidad Politécnica de Madrid, Madrid.
\end{thebibliography}

\end{document}