\documentclass[a4paper,12pt,twoside]{report}

% MAIN CONFIGURATION
\usepackage[top = 2.5cm, bottom = 3cm, left = 4cm, right = 2cm]{geometry}

\usepackage{algorithm,algorithmic}
\usepackage{graphicx}
\usepackage[hidelinks]{hyperref}
\usepackage{indentfirst}
\usepackage{lipsum}
\usepackage[T1]{fontenc} % Font type.
\usepackage{titlesec}
\titleformat{\chapter}[display]{\normalfont\bfseries}{}{0pt}{\Huge}
\usepackage[utf8]{inputenc}
\setlength{\parskip}{0.5cm}
\titlespacing{\paragraph}{0pt}{0pt}{0.1cm}[]

% CONTENT
\begin{document}
  \pagenumbering{Roman}
  \begin{titlepage}
    \newcommand{\HRule}{\rule{\linewidth}{0.5mm}}
    \begin{center}
      \includegraphics[width = 2.25cm]{resources/FACINFO}
      \hspace{8cm}
      \includegraphics[width = 2cm]{resources/logoupm.png}
      \\[1cm]

      \textsc{\Large Escuela Técnica Superior de Ingenieros Informáticos}
      \\[0.5cm]
      \textsc{\large Universidad Politécnica de Madrid}
      \\[3cm]

      \HRule \\[0.4cm]
      {\huge \bfseries Sistema evolutivo híbrido para la construcción de redes de neuronas}
      \HRule \\[4cm]
    
      \textsc{\LARGE Trabajo Fin de Máster}\\[0.5cm]
      \textsc{\Large Máster Universitario en Inteligencia Artificial}\\[3cm]
    \end{center}
    \begin{flushright}
      \large AUTOR: Carlos Vázquez Losada \\TUTOR: Daniel Manrique Gamo
      \\[2.1cm]
    \end{flushright}
    \begin{center}
      {{20 de julio de 2019}}
    \end{center}
    \vfill
  \end{titlepage}
  \newpage\cleardoublepage

  \chapter{\vspace{-3cm}{\LARGE Agradecimientos}}
  \vspace{-1cm}
    Deseo expresar mi agradecimiento, en primer lugar, a Daniel, un investigador incansable y un profesor excepcional, pionero en Computación Evolutiva en España y protagonista indudable de la evolución de la Programación Genética. Gracias por dedicarme tu tiempo y adaptarte a mis horarios, y gracias por despertar en mi el interés en la Computación Evolutiva con tus clases y con tu implicación.\par
    A mis padres, Isabel Losada y Fco. Javier Vázquez, por su amor, comprensión y por su apoyo incondicional en todo aquello que me he propuesto.\par
    A Cristina, espectadora de mis éxios y mi acompañante. Siempre sabes que decir y cómo complementarme. Te quiero.
  \vfill
  \newpage\cleardoublepage
  
  \chapter{\vspace{-3cm}{\LARGE Resumen}}
  \vspace{-1cm}
  Este Trabajo de Fin de Máster consiste en un estudio de la Programación Genética Guiada por Gramáticas y su utilización para la creación de la mejor Red de Neuronas obtenida de un problema determinado, realizando para ello experimentos de entrenamiento total y parcial de cada una de las redes candidatas. \par
  La Programación Genética es una técnica evolutiva (y por tanto, inspirada en la biología) y que es utilizada en problemas de optimización cuyas soluciones son programas informáticos. La Programación Genética Guiada por Gramáticas extiende las posibilidades de la Programación Genética tradicional con la introducción de las gramáticas, que permiten crear individuos sintácticamente válidos. \par
  Estas gramáticas permiten la confección de arquitecturas de Redes de Neuronas que son válidas, dada cualquier número de neuronas en la capa de entrada y en la capa de salida. El resultado de una ejecución nos devolverá la arquitectura de red que mejor se amolde al problema dado, teniendo en cuenta las propias peculiaridades de la gramática utilizada. Además, se estudia si el resultado es el mismo entrenando individuos de forma total o parcial.
  \vfill
  \newpage\cleardoublepage
  
  \chapter{\vspace{-3cm}{\LARGE Summary}}
  \vspace{-1cm}
  This Master Thesis Dissertation consists in a research about Grammar-Guided Genetic Programming and its use to create the best Neural Network given a concrete problem through experiments for both fully and partially trained each neural.
  \vfill
  \newpage\cleardoublepage
  
  \tableofcontents
  \vfill
  \newpage\cleardoublepage
  \pagenumbering{arabic}
  
  \chapter{\vspace{-3cm}{\LARGE 1. Introducción}}
  \vspace{-1cm}
  La optimización matemática 
  
  \chapter{\vspace{-3cm}{\LARGE 2. Computación Evolutiva}}
  
  \chapter{\vspace{-3cm}{\LARGE 3. Redes de Neuronas Artificiales}}
  
  \chapter{\vspace{-3cm}{\LARGE 4. Construcción de Redes de Neuronas}}
  
  \chapter{\vspace{-3cm}{\LARGE 5. Planteamiento del problema}}
  
  \chapter{\vspace{-3cm}{\LARGE 6. Solución propuesta}}
  
  \chapter{\vspace{-3cm}{\LARGE 7. Resultados}}
  
  \chapter{\vspace{-3cm}{\LARGE 8. Soluciones y líneas futuras}}
\end{document}
